% Options for packages loaded elsewhere
\PassOptionsToPackage{unicode}{hyperref}
\PassOptionsToPackage{hyphens}{url}
%
\documentclass[doc]{apa7}

\usepackage{amsmath,amssymb}
\usepackage{lmodern}
\usepackage{iftex}
\ifPDFTeX
  \usepackage[T1]{fontenc}
  \usepackage[utf8]{inputenc}
  \usepackage{textcomp} % provide euro and other symbols
\else % if luatex or xetex
  \usepackage{unicode-math}
  \defaultfontfeatures{Scale=MatchLowercase}
  \defaultfontfeatures[\rmfamily]{Ligatures=TeX,Scale=1}
\fi
% Use upquote if available, for straight quotes in verbatim environments
\IfFileExists{upquote.sty}{\usepackage{upquote}}{}
\IfFileExists{microtype.sty}{% use microtype if available
  \usepackage[]{microtype}
  \UseMicrotypeSet[protrusion]{basicmath} % disable protrusion for tt fonts
}{}
\makeatletter
\@ifundefined{KOMAClassName}{% if non-KOMA class
  \IfFileExists{parskip.sty}{%
    \usepackage{parskip}
  }{% else
    \setlength{\parindent}{0pt}
    \setlength{\parskip}{6pt plus 2pt minus 1pt}}
}{% if KOMA class
  \KOMAoptions{parskip=half}}
\makeatother
\usepackage{xcolor}
\IfFileExists{xurl.sty}{\usepackage{xurl}}{} % add URL line breaks if available
\IfFileExists{bookmark.sty}{\usepackage{bookmark}}{\usepackage{hyperref}}
\hypersetup{
  pdftitle={Can human values explain one's interest in cryptocurrencies? A correlational study in Germany},
  pdfkeywords={Values, Cryptocurrencies, Germany},
  hidelinks,
  pdfcreator={LaTeX via pandoc}}
\urlstyle{same} % disable monospaced font for URLs
\usepackage{longtable,booktabs,array,caption}
\usepackage{graphicx}
\makeatletter
\def\maxwidth{\ifdim\Gin@nat@width>\linewidth\linewidth\else\Gin@nat@width\fi}
\def\maxheight{\ifdim\Gin@nat@height>\textheight\textheight\else\Gin@nat@height\fi}
\makeatother
% Scale images if necessary, so that they will not overflow the page
% margins by default, and it is still possible to overwrite the defaults
% using explicit options in \includegraphics[width, height, ...]{}
\setkeys{Gin}{width=\maxwidth,height=\maxheight,keepaspectratio}
% Set default figure placement to htbp
\makeatletter
\def\fps@figure{htbp}
\makeatother
\setlength{\emergencystretch}{3em} % prevent overfull lines
\providecommand{\tightlist}{%
  \setlength{\itemsep}{0pt}\setlength{\parskip}{0pt}}
\setcounter{secnumdepth}{-\maxdimen} % remove section numbering
\usepackage{subfig}
\usepackage{booktabs}
\usepackage{longtable}
\usepackage{array}
\usepackage{multirow}
\usepackage{wrapfig}
\usepackage{float}
\usepackage{colortbl}
\usepackage{pdflscape}
\usepackage{tabu}
\usepackage{threeparttable}
\usepackage{threeparttablex}
\usepackage[normalem]{ulem}
\usepackage{makecell}
\usepackage{xcolor}
\ifLuaTeX
  \usepackage{selnolig}  % disable illegal ligatures
\fi

\title{Can human values explain one's interest in cryptocurrencies? A
correlational study in Germany}
\shorttitle{Values and cryptocurrencies}

\authorsnames[{1},{1},{2},,{1}]{Adrian Stanciu,Melanie Partsch,Mariana
Bernardes da Silva Passos,Ranjit Singh,Clemens Lechner}
\authorsaffiliations{{Data and Research on Society, GESIS-Leibniz
Institute for the Social Sciences, Mannheim}
,{Data and Research on Society, GESIS-Leibniz Institute for the Social
Sciences, Mannheim}
,{University of Bremen, Bremen}
,{Data and Research on Society, GESIS-Leibniz Institute for the Social
Sciences, Mannheim}
,{Data and Research on Society, GESIS-Leibniz Institute for the Social
Sciences, Mannheim}
}
\authornote{For correspondence contact Dr.~Adrian Stanciu, Data and
Research on Society, GESIS-Leibniz Institute for the Social Sciences, PO
Box 12215, 68072 Mannheim, Germany. Email:
adrian.stanciu{[}at{]}gesis.org}
\date{20. Mai, 2022}
\abstract{Write abstract here}
\keywords{Values, Cryptocurrencies, Germany}




\begin{document}
\maketitle

\begin{center}\rule{0.5\linewidth}{0.5pt}\end{center}

\hypertarget{preamble}{%
\section{Preamble}\label{preamble}}

Cryptocurrencies are digital money, and much more. The idea of
cryptocurrency has developed only recently, in 2009, when a person or a
group of people known as Satoshi Nakamoto have launched a novel currency
that was entirely digital and based on a chain of mathematical
algorithms. Their vision was one in which the centralized financial
system as we know it can one day be de-centralized in that financial
transactions will be possible directly between involved parties. The
idea was a simple one: by eliminating the role of banks and governmental
bodies that secure trust in financial transactions, unnecessary taxes
can be eliminated, people can have more control of their finances, as
well as global barriers due to currency conversion might as well be
abolished. This has since entered the public knowledge domain and is now
known as Bitcoin. Several variations of it and a number of alternative
cryptocurrencies later, and it is becoming increasingly clear that the
topic is no longer in the under-ground of our society, but quite the
contrary, it slowly grows into yet another domain where there is divide
between individuals.

This research sets to explore reasons why some people are interested in
this new form of finance whereas other are against it, or they are
skeptical. We focus on the role of human values as possible explanators
and seek to understand associations with an interest and motivation to
invest in cryptocurrencies among the general German population.

We look at the associations between the ten values and the four higher
order orientations postulated in the Schwartz theory and varying
cryptocurrency items.

TO DO LIST:

\begin{itemize}
\tightlist
\item
  table with constructs and items (pvq dimensions; crypto-items)
\item
  which constructs/items can be treated as continuous, which must be
  treated as ordinal
\end{itemize}

\hypertarget{hypotheses}{%
\section{Hypotheses}\label{hypotheses}}

This study explores the present topic and it does not test for causal
mechanisms. As such, the present hypotheses have the role of guiding our
interpretations of findings, rather than informing claims about theory.

We can think of two approaches to how human values contribute to a
person's interest in crypto-currencies. Please note that interest in
crypto-currencies is here meant in a general sense which addresses both
awareness, currency-holding, as well as overall beliefs about the topic.

First, we acknowledge the fact that crypto-currencies are a new form of
finance thus it is potentially linked with people's motivational goals
of accepting or rejecting novelty in their life. The hypothesis is that
values of openness to change will associate with a stronger interest in
crypto-currencies.

Second, we acknowledge the primary goal of crypto-currencies - that of
creating a decentralized financial system - thus facilitating financial
independence in people who otherwise are dependent on monetary
institutions such as banks. We reason that this might be linked with
people's motivational goal of self-enhancing themselves, striving for
own financial well-being that is. The hypothesis is that values of
self-enhancement will associate with a stronger interest in
crypto-currencies.

Whay should it matter if values inform people's interst in
cryptocurrencies? Values are beliefs that guide action in people, and
this holds across cultures and contexts. What makes values ultimately
relevant is that they are culture-informed, have a developmental
trajectory that is lifelong, while they they can change due to life
events such as entering parenthood, but also due to drastic changes in
society such as due to the Covid-19 Pandemic.

Cryptocurrencies hold several similarities with previous recent
technological advancements in society such as the onset of the internet
and of mobile phones. But, compared to previous advancements the notion
of cryptocurrencies is also in a way an economic and political
statement. Since this technology is still in its onset phase, but has
already shown signs of becoming an established part of our society,
there are reasons to expect that it will impact people's financial
independence and potentially people's dependencies on established
monetary and governmental institutions.

In other words, the concept of cryptocurrency is potentially going to
shape in the future in currently unknown ways people's motivational
goals. We speak of the future because we want to raise awareness that a
new technology is now being adopted, in spite of groups of people being
against it or disbelieving its mission. This is happening. Therefore, to
begin understanding how this will shape our beliefs in the future, we
reason that the pertinent first step is to explore what makes this
technology possible from a social perspective. That is, why some people
are interested, engaged, and open for this form of finance, while others
are not.

\begin{center}\rule{0.5\linewidth}{0.5pt}\end{center}

\hypertarget{data-prepartion}{%
\section{Data prepartion}\label{data-prepartion}}

\hypertarget{participants}{%
\subsection{Participants}\label{participants}}

Overall there are \textbf{N = 692} participants who have at least heard
of crypto-currencies. This is our sample for the study.

\begin{table}

\caption{\label{tab:unnamed-chunk-1}Sample description}
\centering
\begin{tabular}[t]{l|l|r|r|r|r}
\hline
name & value & n & total\_n & prop & cum\\
\hline
edu\_fct & [1] Grundschule nicht beendet & 1 & 692 & 0.14 & 0.14\\
\hline
edu\_fct & [2] (noch) kein Abschluss, aber Grundschule beendet & 9 & 692 & 1.30 & 1.45\\
\hline
edu\_fct & [3] Volks- oder Hauptschulabschluss bzw. Polytechnische Oberschule der ehem. DDR mit Abschluss der 8. oder 9. Klasse & 156 & 692 & 22.54 & 23.99\\
\hline
edu\_fct & [4] Mittlere Reife, Realschulabschluss, Fachoberschulreife, mittlerer Schulabschluss bzw. Polytechnische Oberschule der ehem. DDR mit Abschluss der 10. Klasse & 240 & 692 & 34.68 & 58.67\\
\hline
edu\_fct & [7] Fachhochschulreife (Abschluss einer Fachoberschule etc.) & 52 & 692 & 7.51 & 66.18\\
\hline
edu\_fct & [8] Abitur, allgemeine oder fachgebundene Hochschulreife bzw. Erweiterte Oberschule der ehem. DDR mit Abschluss der 12. Klasse & 234 & 692 & 33.82 & 100.00\\
\hline
employ\_fct & [1] angestellt & 431 & 692 & 62.28 & 62.28\\
\hline
employ\_fct & [2] selbstständig & 27 & 692 & 3.90 & 66.18\\
\hline
employ\_fct & [3] arbeitslos und arbeitssuchend & 33 & 692 & 4.77 & 70.95\\
\hline
employ\_fct & [4] arbeitslos und nicht arbeitssuchend & 9 & 692 & 1.30 & 72.25\\
\hline
employ\_fct & [5] Hausfrau/Hausmann & 45 & 692 & 6.50 & 78.76\\
\hline
employ\_fct & [6] Schüler/Student & 36 & 692 & 5.20 & 83.96\\
\hline
employ\_fct & [7] Auszubildender/Praktikant & 10 & 692 & 1.45 & 85.40\\
\hline
employ\_fct & [8] Rentner & 86 & 692 & 12.43 & 97.83\\
\hline
employ\_fct & [9] nichts vom oben Genannten & 15 & 692 & 2.17 & 100.00\\
\hline
sex\_fct & [1] männlich & 349 & 692 & 50.43 & 50.43\\
\hline
sex\_fct & [2] weiblich & 343 & 692 & 49.57 & 100.00\\
\hline
\end{tabular}
\end{table}

\hypertarget{correlations}{%
\subsection{Correlations}\label{correlations}}

\includegraphics{krypto_md_files/figure-latex/unnamed-chunk-3-1.pdf}

\begin{center}\rule{0.5\linewidth}{0.5pt}\end{center}

\hypertarget{descriptives}{%
\section{Descriptives}\label{descriptives}}

\hypertarget{scale-reliabilities}{%
\subsection{Scale reliabilities}\label{scale-reliabilities}}

\includegraphics{krypto_md_files/figure-latex/unnamed-chunk-4-1.pdf}

\hypertarget{pca-financial-and-techonological-literacy-items}{%
\section{PCA: Financial and techonological literacy
items}\label{pca-financial-and-techonological-literacy-items}}

Financial and technological literacy of participant was measured with 7
items as used in the OECD Consumer Insights Survey on Cryptoassets
(\href{https://www.oecd.org/financial/education/consumer-insights-survey-on-cryptoassets.pdf}{see
here}).

\begin{table}

\caption{\label{tab:unnamed-chunk-5}List of items measuring financial and technological literacy.}
\centering
\begin{tabular}[t]{l|l}
\hline
var\_name & variable\_label\\
\hline
kryftl01 & Ich bin bereit etwas meines eigenen Geldes zu riskieren, wenn ich es anlege oder investiere.\\
\hline
kryftl02 & Ich bin mit meiner aktuellen finanziellen Situation zufrieden.\\
\hline
kryftl03 & Ich lebe im Hier und Jetzt und mache mir wenig Sorgen um Morgen.\\
\hline
kryftl04 & Ich bevorzuge Finanzdienstleister, die sich ethisch verhalten.\\
\hline
kryftl05 & Es macht mir Spaß, mich mit neuen Technologien und ihren Möglichkeiten zu beschäftigen.\\
\hline
kryftl06 & Ich kenne mich gut mit dem Thema Finanzen aus.\\
\hline
kryftl07 & Aufgrund meiner finanziellen Situation bleiben mir viele Dinge im Leben verwehrt.\\
\hline
\end{tabular}
\end{table}

\hypertarget{loadings}{%
\subsection{Loadings}\label{loadings}}

A Principal Component Analysis is conducted on the items measuring
financial and technological literacy.

\includegraphics{krypto_md_files/figure-latex/unnamed-chunk-6-1.pdf}

Structure.

\begin{table}

\caption{\label{tab:unnamed-chunk-7}Factor loadings for financial and technological literacy}
\centering
\begin{tabular}[t]{l|r|r}
\hline
  & RC1 & RC2\\
\hline
kryftl01 & 0.684 & 0.196\\
\hline
kryftl02 & 0.280 & 0.805\\
\hline
kryftl03 & 0.324 & 0.085\\
\hline
kryftl04 & 0.527 & -0.061\\
\hline
kryftl05 & 0.737 & -0.026\\
\hline
kryftl06 & 0.701 & 0.294\\
\hline
kryftl07 & -0.053 & 0.886\\
\hline
\end{tabular}
\end{table}

These plots depict the role of the 2 PC in explaining the variation in
answers as per grouping variables.

\hypertarget{gender}{%
\subsubsection{Gender}\label{gender}}

\includegraphics{krypto_md_files/figure-latex/unnamed-chunk-8-1.pdf}

\hypertarget{education}{%
\subsubsection{Education}\label{education}}

\includegraphics{krypto_md_files/figure-latex/unnamed-chunk-9-1.pdf}

\hypertarget{holds-cryptocurrencies}{%
\subsubsection{Holds cryptocurrencies}\label{holds-cryptocurrencies}}

\includegraphics{krypto_md_files/figure-latex/unnamed-chunk-10-1.pdf}

\hypertarget{holds-cryptocurrencies-in-the-future}{%
\subsubsection{Holds cryptocurrencies in the
future}\label{holds-cryptocurrencies-in-the-future}}

\includegraphics{krypto_md_files/figure-latex/unnamed-chunk-11-1.pdf}

Accordingly, two factors are construed:

\begin{itemize}
\tightlist
\item
  \textbf{Dim1=``financial opportunism''} =
  \texttt{kryftl01,kryftl05,kryftl06}
\item
  \textbf{Dim2=``financial satisfaction''} =
  \texttt{kryftl02,\ kryftl07}
\end{itemize}

Scale reliabilities are presented here.

\begin{table}

\caption{\label{tab:unnamed-chunk-12}Cronbach alpha for financial satisfaction and opportunism}
\centering
\begin{tabular}[t]{l|r|r|r}
\hline
indexe & alpha & ci\_low & ci\_high\\
\hline
financial\_satisfaction & 0.67 & 0.60 & 0.72\\
\hline
financial\_opportunism & 0.66 & 0.58 & 0.70\\
\hline
\end{tabular}
\end{table}

\begin{center}\rule{0.5\linewidth}{0.5pt}\end{center}

To do list:

\begin{itemize}
\tightlist
\item
  set na == DONE
\item
  combine male and female pvq versions into one item == DONE
\item
  check for reverse coding == DONE
\item
  identify which items belong to which constructs: especially for
  valigo, pvq == DONE
\item
  create the new twelve Schwartz value orientations == DONE
\item
  calculate scale reliability for the new twelve Schwartz value
  orientations for pvq == DONE
\item
  calculate scale reliability == DONE for HOV
\item
  construct ipsatized indices for valigo and/pvq == DONE
\item
  PCA on financial and technological literacy == DONE
\item
  calculate scale reliability for constructs from the PCA on literacy
  items
\item
  construct scales for constructs from the PCA on literacy items
\end{itemize}

\end{document}